\section{Introduction}

In recent years, quantum technologies have influenced several fields, including communication, computing, sensing, and imaging. They all have one thing in common: they utilize the non-classical properties of light to improve state-of-the-art technology beyond its existing limits. \newline
Specifically, in quantum imaging and sensing, the quantum nature of light is exploited, so light is considered not only as electromagnetic waves, but also as single photons. \newline
In classical imaging, system parameters like spatial resolution or sensitivity are constrained by the diffraction and shot noise limit. 
Quantum imaging aims to address some of these constraints by making use of quantum properties of light such as entanglement, photon-number correlations, and squeezing~\cite{defienneAdvancesQuantumImaging2024,moreauImagingQuantumStates2019}. \newline
Among the available quantum light sources, \acrfull{spdc} is widely used to generate photon pairs that are correlated in various degrees of freedom, including time, space, and polarization~\cite{moreauImagingQuantumStates2019}. \newline
One example of an imaging technique that uses spatially correlated photons is Quantum ghost imaging. In this technique, an image of an object is reconstructed by measuring the intensity correlations between two spatially separated light beams produced by \acrshort{spdc}. One beam interacts with the sample but is detected with a bucket detector while no spatial information is recorded. The other beam, which does not interact with the sample, is measured with a spatially resolving detector. The spatial information of the sample is recovered by considering the spatial correlation between the two beams \cite{gilabertebassetPerspectivesApplicationsQuantum2019,lemosQuantumImagingUndetected2014}. \newline
Another imaging technique uses time correlation of the photon pairs.
These photons are particularly important in detection schemes based on coincidence measurements. One photon of the pair is sent through a sample while the other is detected directly. Coincidence measurements between the two detection events can then identify photons that were truly part of a pair. This suppresses uncorrelated background noise \cite{moreauImagingQuantumStates2019}. This approach will be referred to as the coincidence approach in the following. It will be compared with the scenario in which only one photon of the pair is sent through the sample, with the second photon being neglected. Since this represents a classical transmission experiment, it will be referred to as the conventional approach. It has been shown that the coincidence approach has advantages in terms of precision over the conventional one \cite{sabines-chesterkingSubShotNoiseTransmissionMeasurement2017,sabines-chesterkingTwinbeamSubshotnoiseRasterscanning2019}. \newline
For certain biological samples, such as living cells, it is necessary that the wavelength of photons is in the mid-infrared (MIR) to ensure a non-destructive interaction with the sample. This wavelength range corresponds to fundamental vibrational absorption bands \cite{katoLabelfreeVisualizationPhotosynthetic2023,ishiganeLabelfreeMidinfraredPhotothermal2023}. However, a major technical challenge in the MIR regime is the low detection efficiency of the detectors and hence a high level of noise. In particular, thermal background radiation leads to high dark count rates and a low signal-to-noise ratio \cite{SingleQuantumExcellence}. \newline
%\acrfull{snr} 
As mentioned above, it has been shown that in the absence of noise and with an efficient setup using coincidence measurements offers an advantage compared to conventional methods. This work aims to determine if measurements using temporally correlated photon pairs still offer advantages over conventional single-photon experiments in regimes of large noise. \newline
First, the mechanism behind producing correlated photon pairs is explained. Then, a procedure for modeling the precision of the parameter estimation is introduced for both the correlation-based and conventional approaches. Next, the photon statistics are explained, as well as the setup used to verify those statistics experimentally. Furthermore, the experimental results are compared with the theoretical predictions. Finally, different parameter regions are simulated in order to find combinations for which the correlation-based approach is superior using the experimental results.







