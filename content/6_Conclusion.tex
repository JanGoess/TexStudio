\section{Conclusion}
This work has investigated the precision of the transmittance estimation using temporally correlated photon pairs generated via spontaneous parametric down-conversion (SPDC) and compared it to a conventional single-photon approach. The central objective was to determine whether the coincidence-based detection scheme maintains an advantage in regimes dominated by noise. \newline
To obtain a quantitative model of each method’s precision, analytical expressions for the variance of the transmittance were derived through error propagation. To ensure that the models reflect realistic conditions, the photon statistics were investigated experimentally. \newline
The experimental results revealed that single-photon statistics in both the idler and signal arms follow a \acrshort{mmbe} distribution, confirming their super-Poissonian character. In contrast, coincidence and accidental counts exhibited Poissonian statistics, validating the theoretical assumption. Based on the experimental results and the analytical expressions for the variance of the transmittance a numerical simulation model was created. \newline
The simulation analyses investigated the dependence of the transmittance variance on key experimental parameters such as the signal arm efficiency, photon rate, and noise-to-signal ratio. The results showed that in noise-dominated regimes, i.e. when the noise-to-signal ratio is considerably large, the coincidence approach yields a lower variance and thus a higher precision than the conventional method. This advantage arises from the ability of coincidence detection to discriminate true photon-pair events from uncorrelated background noise. Additionally, the analysis demonstrated that the coincidence approach becomes more favorable with increasing signal arm efficiency, while at high photon rates and low noise levels both methods perform comparably. \newline
Overall, the simulation confirms that coincidence-based detection of correlated photon pairs provides an advantage in precision under noisy conditions. The presented model predicts several parameter regions in which this advantage becomes significant. \newline Consequently, the next step is to verify the results obtained by the simulation in an experimental setup. In particular, the experiment must demonstrate whether the variance-based approach used in the theoretical model accurately describes the statistical behavior of the measured transmittance. Since the simulation relies on assumed statistical distributions and linear error propagation, it cannot independently confirm the validity of these assumptions. Therefore, only experimental data can reveal to what extent the modeled variances truly represent the physical fluctuations of the measured quantities. \newline
Furthermore, it must be demonstrated to what extent the variance of transmittance can be measured experimentally, and whether it is possible to measure up to the second decimal place. The maximum achievable precision in the experiment also limits the applicable parameter ranges in the simulation to determine the conditions under which the coincidence approach offers advantages in terms of precision. \newline
Another aspect to consider is that, in regimes with a high noise-to-signal ratio, it may be difficult to distinguish accidental and coincidental counts in the histogram for low exposure times. The experiment must show whether increasing exposure time increases the difference between the two. 


\todo[inline]{theoretical mode value}