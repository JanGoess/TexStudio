\section{Introduction2}

In recent years, quantum technologies have influenced several fields, including communication, computing, sensing, and imaging. They all have one thing in common: they utilize the non-classical properties of particles to improve state-of-the-art technology beyond its existing limits. \newline
Specifically, in quantum imaging and sensing, the quantum nature of light is exploited, so light is considered not only as electromagnetic waves, but also as single photons. \newline
In classical imaging, constraints such as the diffraction limit and shot noise limit system parameters like spatial resolution, sensitivity, and required illumination intensity. 
Quantum imaging aims to address some of these constraints by making use of quantum correlations, such as entanglement, photon-number correlations, and squeezing~\cite{defienneAdvancesQuantumImaging2024,moreauImagingQuantumStates2019}. \newline
Among the available quantum light sources, \acrfull{spdc} is widely used to generate photon pairs that are correlated in various degrees of freedom, including time, space, and polarization~\cite{moreauImagingQuantumStates2019}. \newline
One example of an imaging technique that uses spatially correlated photons is Quantum ghost imaging. In this technique, an image of an object is reconstructed by measuring the intensity correlations between two spatially separated light beams produced by \acrshort{spdc}. One beam interacts with the sample but is detected without spatial information. The other beam, which does not interact with the sample, is measured with a spatially resolving detector. The spatial information about the sample is recovered by considering the spatial correlation between the two photon beams \cite{gilabertebassetPerspectivesApplicationsQuantum2019,lemosQuantumImagingUndetected2014}. \newline
Another imaging technique uses temporally correlated photons.
These photons are particularly important in detection schemes based on coincidence measurements. For example, in a transmission setup, temporally correlated photon pairs can be used to estimate the characteristic properties of a sample. \newline
One photon of the pair is sent through the sample while the other is detected directly. Coincidence measurements between the two detection events identify photons that were truly part of a pair. This suppresses uncorrelated background noise ~cite{moreauImagingQuantumStates2019}. \newline
It has been shown that this approach has advantages over the conventional method, in which only photons passing through the sample are detected, and correlations are not considered [references???].
For certain biological samples, such as living cells, it is necessary that the wavelength of photons is in the mid-infrared (MIR) to ensure a non-destructive interaction with the sample. This wavelength range corresponds to fundamental vibrational absorption bands. \newline
However, a major technical challenge in the MIR regime is that the detectors have a low sensitivity and the level of noise is significantly high. In particular, thermal background radiation lead to high dark count rates and low signal-to-noise ratios (SNRs) [SNSPD/SPD reference]. \newline
This work aims to determine if measurements using temporally correlated photon pairs still offer advantages over conventional single-photon experiments in regimes of large noise. First, the mechanism behind producing correlated photon pairs is explained. Then, a procedure for modeling the precision of transmission is introduced for both the correlation-based and conventional approaches. Next, the statistics of the necessary parameters are explained, as well as the setup used to verify those statistics experimentally. Furthermore, the experimental results are compared with the theoretical predictions. Finally, different parameter regions are simulated in which the correlation-based approach is superior using the experimental results.

